\documentclass[a4paper,12pt,openright,oneside]{report}

\usepackage[utf8]{inputenc}
\usepackage[portuguese]{babel} 
\usepackage{graphicx}
\usepackage{hyperref} 
\usepackage[T1]{fontenc}

\begin{document}
\begin{titlepage}
\begin{center}
\includegraphics[width=0.40\textwidth]{logo_UA_whitebg.png}~\\[1cm]
\begin{large}
Engenharia de Computadores e Telemática\\[0.5cm]
Laboratório de Sistemas Digitais
\end{large}

\vspace{4cm}
\begin{LARGE}
{\Huge \textbf{\textit{Relatório:} Euromilhões}}
\end{LARGE}
\vspace{5.5cm}

\begin{minipage}{0.4\textwidth}
\begin{flushleft} \large
\begin{large}
\textbf{Autores:}\\
Hugo Fragata\\
Gonçalo Alexandre\\[0.2cm]
\end{large}
\begin{footnotesize}
\textbf{}
\end{footnotesize}
\end{flushleft}
\end{minipage}

\vfill
{\normalsize  2 Junho 2014}
\end{center}
\end{titlepage}

 \chapter*{Resumo}
 \paragraph*{}
Neste relatório é feita uma análise pormenorizada do formato e estrutura da página Web, cujo endereço é http://xkcd.com. Esta página foi concebida através de um documento HTML que se apresenta divido em blocos, como qualquer outro documento HTML. Esta divisão é útil para posicionar o conteúdo de acordo com a sua função e, consequentemente, melhorar a apresentação do conteúdo.
\paragraph*{}
No conteúdo da página Web, referente ao elemento <body>, coexistem três grandes divisões, Top Container, Middle Container e Bottom Container.
\subparagraph*{}
Cada bloco é analisado separadamente, de forma a identificar e descrever a estrutura e os elementos HTML utiizados em cada divisão. Elementos esses que definem a categoria do conteúdo, por exemplo, listas de itens, tabelas, cabeçalhos, hiperligações, entre outros. A análise concentra-se no formato e ignora  estilo utilizado pelo autor.  


 \addcontentsline{toc}{chapter}{Resumo}
 
 
\chapter*{Glossário}
 \addcontentsline{toc}{chapter}{Glossário}
\begin{enumerate} 
\renewcommand{\theenumi}{\Roman{enumi}} 
\item \textbf{CSS:} Cascading Style Sheets
\item \textbf{Favicon:} Icon da página
\item \textbf{Tag:} Palavra que define o elemento
\item \textbf{Browser:} Navegador da internet
\item \textbf{URL:} Uniform Resource Locator (é o endereço)
\item \textbf{Metadata:} Dados acerca de dados
\end{enumerate}

\renewcommand{\contentsname}{Índice} 
\tableofcontents
\listoffigures

\chapter*{Introdução}
\paragraph*{}
HTML, abreviação de HyperText Markup Language, é uma linguagem de marcação, ou seja, um conjunto de regras e códigos que se inserem num arquivo de texto simples para representação da informação pretendida, que é usado para criar documentos na World Wide Web.
\paragraph*{}
Um documento HTML é um documento de texto com marcas, tags, que definem a sua estrutura e formatação. Todos os elementos possuem início e podem apresentar fim e conteúdo, cuja função é identificada por esse elemento. Além disso podem apresentar atributos que clarificam o elemento. Estes tags dão indicações ao navegador de como deve interpretar os conteúdos, que serão posteriormente apresentados ao utilizador.\cite{2site}
\subparagraph*{}
Todos os documentos HTML seguem uma estrutura rígida pois começam com as marcas <html> e <head>, cujo conteúdo engloba o título e definições de estilo e scripts, </head> <body>, e termina com </body> </html>. Toda a informação e conteúdo da página Web apresentável ao utilizador está delimitada pelo <body> e </body>.\cite{1site}
\paragraph*{}
O objectivo principal deste relatório é decompor a página relativa ao endereço http://xkcd.com, identificando os seus blocos, qual a estrutura interna de cada um, e como são utilizadas as marcas mais relevantes.
\paragraph*{}
O conteúdo do relatório está dividido em três capítulos: o primeiro está direccionado para a estrutura geral do código referente à página Web, o segundo evidencia a informação presente no elemento HTML <head>, e o terceiro é relativo ao elemento HTML <body>.
\subparagraph*{}
É no terceiro capítulo que estão devidamente identificados os blocos encontrados no documento HTML e a sua respectiva análise e descrição detalhada. Cada secção deste capítulo engloba uma das divisões.
\addcontentsline{toc}{chapter}{Introdução}


\chapter{Estrutura geral do documento HTML}
\section*{}
\begin{figure}[h!]
\center
\includegraphics[scale=0.3]{Desenho1.png}
\caption{Estrutura do código}
\label{Desenho1}
\end{figure}

\paragraph{}
Os elementos de maior importância para funcionamento da leitura de ficheiro HTML são o <html>, o <head> e o <body>.
Nesta página a ser analisada temos a estrutura correcta do ponto de vista da documentação e sintaxe do HTML. Em primeiro lugar é aberto o elemento <html> e,  continuamente, é aberto o elemento <head>. O <head> é então fechado com </head> e, logo de seguida, é aberto o elemento <body>. Após isso, o elemento <body> é fechado com </body> e também é fechado o <html> com </html>. Esta estrutura pode ser observada na figura \ref{asas}. \cite{b}

\begin{figure}[h!]
\center
\includegraphics[scale=0.7]{asas.png}
\caption{Estrutura base comum}
\label{asas}
\end{figure}


\chapter{Head}
\paragraph{}
O <head> é o elemento resposável por definir o título, o favicon, a codificação do caracteres usados, o estilo da página em CSS e possíveis programas de javascript. Nada do que está no elemento <head> aparece no corpo da página, ou seja, não é directamente vísivel ao utilizador.
\subparagraph{}
O elemento <head> deve seguir-se logo após do elemento <html>, que inicia o código da página. Deve incluir, dentro dele, o elemento <title>, responsável pelo título da página.
\paragraph{}
No <head> da página xkcd.com encontramos a tag <title>, o título da mesma, seguida da tag </title>. Isto define o título da página. 
O <head> também possui um conjunto de tags <meta>, que definem metadata sobre a página HTML. Metadata que inclui o tipo de codificação de caracteres, os autores, em que linguagem se encontra e informações para os browsers. 
\paragraph{}
O conjunto de tags <link> também aqui se encontra e serve para definir uma relação entre o documento HTML e recursos externos a este, tais como CSS ou JavaScript.
Tem-se, por fim, a tag <script> que serve para chamar um código em JavaScript externo ou definir directamente tal código.
A figura \ref{awaw} representa a estrutura do elemento <head> na página xkcd.com. \cite{Guião}
 
\begin{figure}[h!]
\center
\includegraphics[scale=0.8]{awaw.png}
\caption{Estrutura do elemento <head>}
\label{awaw}
\end{figure}

\chapter{Body}
\section{Estrutura geral do body}
\paragraph{}
O body, em sentido figurativo, o corpo da página, é aquilo é directamente vísivel ao utilizador. É o conteúdo da página em si, enquanto que o <head> contém definições sobre tal. Contém texto, links, tabelas, imagens, divisoẽs e barras de procura. 
\paragraph{}
O <body> da página xkcd.com está dividido em três blocos, ou divisões, com a ajuda dos elementos <div>. 
\subparagraph{}
A divisão do topo, que inclui uma tabela com alguns links e uma apresentação com imagem. De seguida tem-se a divisão do meio, que inclui o comic e botões de navegação. Por fim, tem-se a divisão do fundo, que inclui umas imagens, uma barra de procura e alguns links e mensagens dos desenvolvedores. Tal estrutura pode ser evidenciada pela figura \ref{ferg}.\cite{PP}

\begin{figure}[h!]
\center
\includegraphics[scale=0.5]{ferg.png}
\caption{Estrutura geral do elemento <body>}
\label{ferg}
\end{figure}

\pagebreak
\section{Divisão do topo}
\paragraph{}
A divisão do topo está subdividida em mais duas divisões. A divisão da esquerda e a da direita. Tal estrutura pode ser vista na figura \ref{ion}.

\begin{figure}[p]
\center
\includegraphics[scale=0.5]{ion.png}
\caption{Estrutura da divisão do topo}
\label{ion}
\end{figure}

\paragraph{}
A divisão da esquerda contém apenas uma lista de itens, não numerada, com alguns links. A lista é definida a partir da tag <ul> e fechada com </ul>. Cada intem é iniciado com a tag <li> e fechada com </li>. Todas as <li> devem ser aberta e fechadas depois de <ul> e antes de </ul>. 
\subparagraph{}
Dentro de cada linha tem-se uma hiperligação, que é definida pela tag <a> e fechada com a tag </a>. O URL é definido da seguinte maneira: <a href="URL"></a>. Depois de <a href="URL"> e antes de </a> está o que se quer associar a tal hiperligação. é possível observar a estrutura da subdivisão da esquerda na \ref{werk}, e o resultado na página Web na figura \ref{swag1}.

\begin{figure}[p]
\center
\includegraphics[scale=0.5]{werk.png}
\caption{Bloco topo: divisão da esquerda}
\label{werk}
\end{figure}

\begin{figure}[p]
\center
\includegraphics[scale=0.5] {swag1.png}
\caption{Divisão da esquerda: Página Web}
\label{swag1}
\end{figure}

\paragraph{}
A subdivisão da direita tem mais duas subdivisões dentro dela. A figura \ref{ion} demonstra isso mesmo. Temos a divisão intitulada masthead e a intitulada news. A primeira apenas tem a imagem (figura \ref{terrible}) e o slogan do site. Para definir a imagem é usada a tag <img/>. Para definir a localização da imagem é utilizada a seguinte forma <img src="localização"/>. De seguida tem-se a divisão "news", que é auto-explicatória, tendo apenas dentro dela um pequeno texto e uma hiperligação. O resultado do código está expresso na figura \ref{swag2}.

\begin{figure}[p]
\center
\includegraphics[scale=0.9]{terrible_small_logo.png}
\caption{Imagem do site}
\label{terrible}
\end{figure}

\begin{figure}[h!]
\includegraphics[scale=0.5]{swag2.png}
\caption{Divisão da direita: Página Web}
\label{swag2}
\end{figure}
\pagebreak
\section{Divisão do meio}
\paragraph{}
A estrutura da  divisão do meio, intitulada "middleContainer", é verificada na figura \ref{yme}, e o seu resultado na figura \ref{swag3}
Na subdivisão "ctitle" tem-se o título da página, em texto. Depois, na subdivisão "comicNav" está apresentada, mais uma vez, uma lista de itens não numerada, com a tag <ul>, que contém hiperligações para navegar o site, tais como, desenho anterior ou próximo ou até desenho aleatório. De seguida tem-se a subdivisão "comic", auto-explicatória, tira partido da tag <img/> para mostrar o comic em destaque. Volta, de seguida, a ter-se exactamente a mesma subdivisão "comicNav". Por fim, aparece apenas um pouco de texto a mostrar hiperligações para a imagem destacada.

\begin{figure}[h!]
\center
\includegraphics[scale=0.5]{yme.png}
\caption{Divisão do meio}
\label{yme}
\end{figure}

\begin{figure}[h!]
\center
\includegraphics[scale=0.3]{swag3.png}
\caption{Divisão do meio: Página Web}
\label{swag3}
\end{figure}
\pagebreak
\section{Divisão do fim}
\paragraph{}
Na divisão final, intitulada "bottom", tem-se imagens e texto, cuja estrutura pode ser observada na figura \ref{iogurte} e o seu resultado na figura \ref{swag4}. Dentro do "bottom" estão mais subdivisões do que nas anteriores. 
\subparagraph{}
Para começar tem-se um mapa de imagens, definido com a ajuda da tag <img/> e <map>, que mostra outros comics. Depois tem-se a subdivisão com uma barra de procura, escrita a partir de uma tag <script> e também de um <form>. Tal estrutura pode ser evidenciada na figura \ref{iogurte}. De seguida, é apresentada a subdivisão "comicLinks" que é uma série de hiperligações separadas por parágrafos com a tag <br/>. Tais hiperligações estão de baixo de um título em texto que diz "Comics I Enjoy:". 
\subparagraph{}
Por fim, existem três pequenos textos. O primeiro é aglomerado pela tag de parágrafo <p> e </p>, o segundo pela subdivisão "footnote", que apenas contém texto, e o terceiro pela subdivisão "licenseText" que apenas contém texto com hiperligações.\cite{a}

\begin{figure}[h!]
\center
\includegraphics[scale=0.5]{iogurte.png}
\caption{Divisão final}
\label{iogurte}
\end{figure}

\begin{figure}[h!]
\center
\includegraphics[scale=0.3]{swag4.png}
\caption{Divisão final: Página Web}
\label{swag4}
\end{figure}

\chapter*{Conclusão}
\addcontentsline{toc}{chapter}{Conclusão}
\paragraph*{}
Fez-se uma descrição detalhada da página escolhida, cujo o URL é http://xkcd.com, focada na estrutura do documento e análise dos principais elementos HTML que forem utilizados em cada bloco. 
Como era pedido deu-se ênfase e prioridade à análise da estrutura, em detrimento do estilo aplicado.
\paragraph*{}
O primeiro cpítulo apresenta um diagrama para representar a estrutura básica do documento HTML referente à página Web.
No terceiro capítulo fez-se a análise pormenorizada das três grandes divisões encontradas no conteúdo delineado pelo elemento <body>. Referiu-se marcas importantes que estão evidenciadas, nomeadamente relativas a hiperligações e listas de itens.
\paragraph*{}
Poderia ser interessante analisar o estilo utilizado nesta página, mas estava, explicitamente, fora do âmbito do relatório em questão.



\addcontentsline{toc}{chapter}{Bibliografia}
\bibliographystyle{plain} 
\bibliography{Rel}

\end{document}